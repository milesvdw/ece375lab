% template created by: Russell Haering. arr. Joseph Crop
\documentclass[12pt,letterpaper]{article}
\usepackage{anysize}
\marginsize{2cm}{2cm}{1cm}{1cm}

\begin{document}

\begin{titlepage}
    \vspace*{4cm}
    \begin{flushright}
    {\huge
        ECE 375 Lab 2\\[1cm]
    }
    {\large
        C → Assembler → Machine Code → TekBot
    }
    \end{flushright}
    \begin{flushleft}
    Lab Time: Tuesday 10-12
    \end{flushleft}
    \begin{flushright}
    Mathew Popowski

    Miles Van de Wetering
    \vfill
    \rule{5in}{.5mm}\\
    TA Signature
    \end{flushright}

\end{titlepage}

\section{Introduction}
The purpose of this lab is to gain an understanding of how C can be used to program the board instead of assembly using the AVR software. It provides insight into how C can translate into machine code because there was already an example of the program written in assembly. The new program was written to mimic the effects of the the first lab using C instead of assembly and was downloaded onto the board.


%\section{Internal Register Definitions and Constants}
%Text goes here

%\section{Interrupt Vectors}
%Text goes here

\section{Program Initialization}
The initiealization routine first initializes the stack pointer, then Port B was initialized to all outputs and Port D was initialized to all inputs and the pull-up resistors were initialized. Port B controls the LEDs and Port D recieves the button inputs. The Move Forward command then runs on the bot until it recieves input. 

\section{Main Program}
The main routine runs a simple loop which checks the values of all of the buttons associated with Port D. If no buttons are depressed, the bot moves forwards. If the left button is pressed, or both the left and right buttons are depressed, the bot backs up, turns right, then checks again. If the right button is depressed, the bot backs up, turns left, and checks the buttons again.

%\section{A Subroutine}
%Text goes here

%\section{Stored Program Data}
%Text goes here

\section{Additional Questions}
\begin{enumerate}
    \item
    This lab required you to compile two C programs (one given as a sample,
    and another that you wrote) into a binary representation that allows them to
    run directly on your mega128 board. Explain some of the benefits of writing
    code in a language like C that can be “cross compiled”. Also, explain some
    of the drawbacks of writing this way.

    One of the biggest advantages of writing in a higher-level language like C is that, because individual compilers take care of idiosyncrasies specific to different machines and architectures, the writer of a given program needn't consider them. Rather, they may write a generic program which, when compiled, will have small differences in the binary depending upon where it was compiled. Unfortunately, this method has the drawback of generally creating larger, less efficient code than if programs are specifically designed for individual architectures. This is because code written for cross-platform compilation is inherently unable to take advantage of some of the specific opportunities for optimization available on each architecture.

    \item
    The C program you just wrote does basically the same thing as the sample
    assembly program you looked at in Lab 1. What is the size (in bytes) of
    your Lab 1 \& Lab 2 output .hex files? Can you explain why there is a size
    difference between these two files, even though they both perform the same
    BumpBot behavior?

    The Lab1 hex file is 485 bytes, while the Lab2 hex file is 840 bytes. This is most likely due to inefficiencies in the generic compilation procedures through which a C program must go in order to be cross-platform compatible. A program written directly in assembly code is much more likely to be minimal than a C program compiled to binary because the C program is written to be able to be compiled on many machines, rather than just this particular micro-controller.
    
\end{enumerate}

\section{Difficulties}
The main difficulty in completing the lab was forgetting to use PIND instead of PORTD to get input from the buttons.

\section{Conclusion}
The lab was our first opportunity to program our boards with our own code and see how it behaved. We learned a lot about the board, and how written programs interact with the hardware. It was helpful to have written a program in a familiar language before we move on to assembly for future labs. 

\section{Source Code}
\begin{verbatim}
*
 * Lab2_Original.c
 *
 * Created: 10/3/2016 9:09:36 PM
 * Author : Mathew
 */ 

/*
PORT MAP
Port B, Pin 4 -> Output -> Right Motor Enable
Port B, Pin 5 -> Output -> Right Motor Direction
Port B, Pin 7 -> Output -> Left Motor Enable
Port B, Pin 6 -> Output -> Left Motor Direction
Port D, Pin 1 -> Input -> Left Whisker
Port D, Pin 0 -> Input -> Right Whisker
*/

#define F_CPU 16000000
#include <avr/io.h>
#include <util/delay.h>
#include <stdio.h>

int main(void)
{
	DDRB = 0b11111111;	//Initialize LEDs
	PORTB = 0b11110000; //Set LED values
	
	DDRD = 0b00000000;	//Initialize Buttons
	PORTD = 0b11111111;	//Set Initial Button Values

	while (1) // loop forever
	{
		while (PIND == 0b11111111)		//No buttons pressed
		{
			PORTB = 0b01100000;			//Forward
		}
		
		if (PIND == 0b11111110 || PIND == 0b11111100)		//Left or Both Buttons
		{
			PORTB = 0b00000000;			//Backwards
			_delay_ms(1000);			//Wait 1s
			PORTB = 0b00100000;			//Right
			_delay_ms(500);				//Wait .5s
		}

		if (PIND == 0b11111101)			//Right Button
		{
			PORTB = 0b00000000;			//Backwards
			_delay_ms(1000);			//Wait 1s
			PORTB = 0b01000000;			//Right
			_delay_ms(500);				//Wait .5s
		}
	}
}
\end{verbatim}
\end{document}

% template created by: Russell Haering. arr. Joseph Crop
\documentclass[12pt,letterpaper]{article}
\usepackage{anysize}
\marginsize{2cm}{2cm}{1cm}{1cm}

\begin{document}

\begin{titlepage}
    \vspace*{4cm}
    \begin{flushright}
    {\huge
        ECE 375 Lab 4\\[1cm]
    }
    {\large
        Interrupts 
    }
    \end{flushright}
    \begin{flushleft}
    Lab Time: Tuesday 10-12
    \end{flushleft}
    \begin{flushright}
    Miles Van de Wetering

	Mathew Popowski
    \vfill
    \rule{5in}{.5mm}\\
    TA Signature
    \end{flushright}

\end{titlepage}

\section{Introduction}
In this lab, we learned to work with the interrupt vectors. We created an interrupt detecting a rising edge on the left and right whiskers which triggered turns.

\section{Internal Register Definitions and Constants}
WskrR, WskrL, EngEnR, EngEnL, EngDirR, and EngDirL were constants used to address specific bits in order to more easily control the motors. WTime and BWTime were constants used to calculate wait loop time, and waitcnt, ilcnt, and olcnt were loop counters. mpr was a general use variable.

\section{Program Initialization}
To initialize our program we directed the stack pointer to the end of the program memory section, allowing us to push and pop from the stack as we change scopes (we did very little of this). We then set up Port B and D to listen for input, and started the tekbot moving forward. Lastly, we set up interrupts for the rising edges to turn left and right.

\section{Main Program}
There was effectively no main (one line infinite loop).

\section{HitRight}
This was the function called when the right whisker was hit. It first clears interrupts to prevent nested interrupts, then backs up for one second, turns the bot left for one second, then starts moving forward again. Lastly, it sets the interrupts back up.

\section{HitLeft}
This was the function called when the left whisker was hit. It first clears interrupts to prevent nested interrupts, then backs up for one second, turns the bot right for one second, then starts moving forward again. Lastly, it sets the interrupts back up.

\section{Interrupt Vectors}
As described in the initialization process, we created two interrupt vectors. The first and second external interrupt vectors were set up, one for hit right and one for hit left.

\section{Additional Questions}
\begin{enumerate}
    \item
    As this lab, Lab 1, and Lab 2 have demonstrated, there are always multiple
    ways to accomplish the same task when programming (this is especially
    true for assembly programming). As an engineer, you will need to be able
    to justify your design choices. You have now seen the BumpBot behavior
    implemented using two different programming languages (AVR assembly
    and C), and also using two different methods of receiving external input
    (polling and interrupts).
    Explain the benefits and costs of each of these approaches. Some important
    areas of interest include, but are not limited to: efficiency, speed, cost of
    context switching, programming time, understandability, etc.

	The primary advantage of using assembly over C is that we have much more fine-grained control over the program that is produced. Furthermore, the compiled assembly code from C generally is much larger than the regular assembly that we would write manually, leading to a minor inefficiency. The advantage of course of using C is access to higher level abstractions unavailable in the assembly programming language, and the fact that many programmers are simply more used to programming in C.
	
	The advantage of using polling over interrupts is that they are somewhat easier to set up (in our opinion) and they can be faster in some circumstances. Most of the time however, they cause unnecessary burden on the processor because they are constantly checking for 'hits' which occur only a small fraction of the time. Polling is a fine choice when hits are frequent and regular. Interrupts, on the other hand, don't create this same inefficiency because work is only done when an interrupt occurs. They are best used when hits are irregular or infrequent.
	
	\item
	Instead of using the Wait function that was provided in BasicBumpBot.asm,
	is it possible to use a timer/counter interrupt to perform the one-second
	delays that are a part of the BumpBot behavior, while still using external
	interrupts for the bumpers? Give a reasonable argument either way, and be
	sure to mention if interrupt priority had any effect on your answer.
	
	It seems possible to use interrupts to perform this sort of timing operation, made much more complicated by the fact that this would be a nested interrupt. We'd probably need to set the mask at the beginning of every interrupt function, or make careful use of interrupt priority in order to ensure that the timing doesn't get interrupted by an inappropriate signal.

\end{enumerate}

\section{Difficulties}
We didn't have any major difficulties with this lab.

\section{Conclusion}
This was a pretty interesting lab, a little harder to comprehend this code than usual since thinking about bits while writing vectors in hexadecimal is sometimes hard. Interrupts certainly seem better for this application than polling, and are useful in general.

\section{Source Code}
Base Code:
\begin{verbatim}
;***********************************************************
;*
;*	Lab 6
;*
;*	Interrupts
;*
;***********************************************************
;*
;*	 Author: Mathew & Miles
;*	   Date: 11/7/2016
;*
;***********************************************************

.include "m128def.inc"			; Include definition file

;***********************************************************
;*	Internal Register Definitions and Constants
;***********************************************************
.def	mpr = r16				; Multipurpose register 

.def	waitcnt = r17				; Wait Loop Counter
.def	ilcnt = r18				; Inner Loop Counter
.def	olcnt = r19				; Outer Loop Counter

.equ	WTime = 100				; Time to wait in wait loop
.equ	BWTime = 200			; Time to wait in reverse

.equ	WskrR = 0				; Right Whisker Input Bit
.equ	WskrL = 1				; Left Whisker Input Bit
.equ	EngEnR = 4				; Right Engine Enable Bit
.equ	EngEnL = 7				; Left Engine Enable Bit
.equ	EngDirR = 5				; Right Engine Direction Bit
.equ	EngDirL = 6				; Left Engine Direction Bit

;/////////////////////////////////////////////////////////////
;These macros are the values to make the TekBot Move.
;/////////////////////////////////////////////////////////////

.equ	MovFwd = (1<<EngDirR|1<<EngDirL)	; Move Forward Command
.equ	MovBck = $00				; Move Backward Command
.equ	TurnR = (1<<EngDirL)			; Turn Right Command
.equ	TurnL = (1<<EngDirR)			; Turn Left Command
.equ	Halt = (1<<EngEnR|1<<EngEnL)		; Halt Command

;***********************************************************
;*	Start of Code Segment
;***********************************************************
.cseg							; Beginning of code segment

;***********************************************************
;*	Interrupt Vectors
;***********************************************************
.org	$0000					; Beginning of IVs
rjmp 	INIT			; Reset interrupt

; Set up interrupt vectors for any interrupts being used
.org	$0002
rcall HitRight
reti

.org	$0004
rcall HitLeft
reti
; This is just an example:
;.org	$002E					; Analog Comparator IV
;		rcall	HandleAC		; Call function to handle interrupt
;		reti					; Return from interrupt

.org	$0046					; End of Interrupt Vectors

;***********************************************************
;*	Program Initialization
;***********************************************************
INIT:							; The initialization routine
; Initialize the Stack Pointer (VERY IMPORTANT!!!!)
ldi		mpr, low(RAMEND)
out		SPL, mpr		; Load SPL with low byte of RAMEND
ldi		mpr, high(RAMEND)
out		SPH, mpr		; Load SPH with high byte of RAMEND

; Initialize Port B for output
ldi		mpr, $FF		; Set Port B Data Direction Register
out		DDRB, mpr		; for output
ldi		mpr, $00		; Initialize Port B Data Register
out		PORTB, mpr		; so all Port B outputs are low		

; Initialize Port D for input
ldi		mpr, $00		; Set Port D Data Direction Register
out		DDRD, mpr		; for input
ldi		mpr, $FF		; Initialize Port D Data Register
out		PORTD, mpr		; so all Port D inputs are Tri-State

; Initialize TekBot Forward Movement
ldi		mpr, MovFwd		; Load Move Forward Command
out		PORTB, mpr		; Send command to motors
; Initialize external interrupts

; Set the Interrupt Sense Control to falling edge 
ldi		mpr, $0A
sts		EICRA, mpr
; Configure the External Interrupt Mask
ldi		mpr, $03
out		EIMSK, mpr
; Turn on interrupts
; NOTE: This must be the last thing to do in the INIT function
sei

;***********************************************************
;*	Main Program
;***********************************************************
MAIN:							; The Main program


rjmp	MAIN			; Create an infinite while loop to signify the 
; end of the program.

;***********************************************************
;*	Functions and Subroutines
;***********************************************************

;-----------------------------------------------------------
;	You will probably want several functions, one to handle the 
;	left whisker interrupt, one to handle the right whisker 
;	interrupt, and maybe a wait function
;------------------------------------------------------------

;-----------------------------------------------------------
; Func: Template function header
; Desc: Cut and paste this and fill in the info at the 
;		beginning of your functions
;-----------------------------------------------------------
HitRight:
cli
push	mpr			; Save mpr register
push	waitcnt			; Save wait register
in		mpr, SREG	; Save program state
push	mpr			;

; Move Backwards for a second
ldi		mpr, MovBck	; Load Move Backward command
out		PORTB, mpr	; Send command to port
ldi		waitcnt, BWTime	; Wait for 1 second
rcall	Wait			; Call wait function

; Turn left for a second
ldi		mpr, TurnL	; Load Turn Left Command
out		PORTB, mpr	; Send command to port
ldi		waitcnt, WTime	; Wait for 1 second
rcall	Wait			; Call wait function

; Move Forward again	
ldi		mpr, MovFwd	; Load Move Forward command
out		PORTB, mpr	; Send command to port

pop		mpr		; Restore program state
out		SREG, mpr	;
pop		waitcnt		; Restore wait register
ldi		mpr, $ff
out		EIFR, mpr
pop		mpr		; Restore mpr
sei
ret				; Return from subroutine

;----------------------------------------------------------------
; Sub:	HitLeft
; Desc:	Handles functionality of the TekBot when the left whisker
;		is triggered.
;----------------------------------------------------------------
HitLeft:
cli
push	mpr			; Save mpr register
push	waitcnt			; Save wait register
in		mpr, SREG	; Save program state
push	mpr			;

; Move Backwards for a second
ldi		mpr, MovBck	; Load Move Backward command
out		PORTB, mpr	; Send command to port
ldi		waitcnt, BWTime	; Wait for 1 second
rcall	Wait			; Call wait function

; Turn right for a second
ldi		mpr, TurnR	; Load Turn Left Command
out		PORTB, mpr	; Send command to port
ldi		waitcnt, WTime	; Wait for 1 second
rcall	Wait			; Call wait function

; Move Forward again	
ldi		mpr, MovFwd	; Load Move Forward command
out		PORTB, mpr	; Send command to port

pop		mpr		; Restore program state
out		SREG, mpr	;
pop		waitcnt		; Restore wait register
ldi		mpr, $ff
out		EIFR, mpr
pop		mpr		; Restore 
sei
ret				; Return from subroutine

;----------------------------------------------------------------
; Sub:	Wait
; Desc:	A wait loop that is 16 + 159975*waitcnt cycles or roughly 
;		waitcnt*10ms.  Just initialize wait for the specific amount 
;		of time in 10ms intervals. Here is the general eqaution
;		for the number of clock cycles in the wait loop:
;			((3 * ilcnt + 3) * olcnt + 3) * waitcnt + 13 + call
;----------------------------------------------------------------
Wait:
push	waitcnt			; Save wait register
push	ilcnt			; Save ilcnt register
push	olcnt			; Save olcnt register

Loop:	ldi		olcnt, 224		; load olcnt register
OLoop:	ldi		ilcnt, 237		; load ilcnt register
ILoop:	dec		ilcnt			; decrement ilcnt
brne	ILoop			; Continue Inner Loop
dec		olcnt		; decrement olcnt
brne	OLoop			; Continue Outer Loop
dec		waitcnt		; Decrement wait 
brne	Loop			; Continue Wait loop	

pop		olcnt		; Restore olcnt register
pop		ilcnt		; Restore ilcnt register
pop		waitcnt		; Restore wait register
ret				; Return from subroutine					; End a function with RET

\end{verbatim}
\end{document}
